\chapter{Requirement Analysis}\label{sec:reqs}\minitoc\vspace{.5cm}
\index{Requirements}

% https://www.cs.fsu.edu/~lacher/courses/COP3331/rad.html
\section{Introduction}
The two main objectives of this project are to conduct research and implement the concept of a multipath networking tunnel utilizing the \ac{xdppage} technology. The project will deliver a comprehensive thesis documentation along with a functional implementation, in the form of a test program based on the \ac{MTX} library.
This \ac{MTX} library will enable the establishment of a network tunnel between two machines, allowing traffic to flow simultaneously over multiple network links. 
The test program will serve as a demonstration platform for the library, simulating a transport session using the \ac{gptu} protocol — a 5G technology employed for tunneling a significant volume of non-homogeneous data.
During the demonstration, the library's capabilities will be showcased, including flow management, throughput control, and other essential features that align with the intended purpose of the library.

\section{Proposed solution}\label{sec:reqs:proposed_solution}
The MTX library is proposed, which offers multipath capability with support for non-homogeneous and concurrent usage.
To ensure optimal path utilization and minimal latency, we will utilize \ac{xdppage} technology to handle the movement of tunnel traffic. 
Specifically, the tunnel traffic will consist of UDP packets, allowing for efficient data transmission.

Multipath technique can serve as a means of horizontally scaling networking capacity by aggregating resources to increase connection bandwidth. 
Additionally, it offers a way to enhance reliability by incorporating redundancy into the network connection. Furthermore, multipath can improve quality of service by enabling optimized treatment and dedicated links for different types of data flows.
Finally, through the establishment of a multipath tunnel, the resources available on each path can be monitored and efficiently allocated among concurrent applications. 
This leads to improved resource utilization and overall efficiency in the network.

\begin{figure}[H]
	\centering
	\includegraphics[width=1.0\textwidth]{tunnel_supports_multiple_flows_diff_qos.png}
	\caption{Tunnel supports multiple flows with different QoS over multiple path. The path can function either as a failover path or consistently participate in the transmission process.}\label{fig:reqs:tunnel_supports_multiple_flows_diff_qos}
\end{figure}

\subsection{Functional requirements}
The functional requirements for the \ac{MTX} library shall include ability to:
\begin{itemize}
    \item Establish a tunnel between 2 hosts and exchange data through the tunnel over the chosen \ac{NIC}s.
    \item Send and receive tunnel's UDP packets using \ac{xdppage} socket exclusively.
    \item Support for path-level control and treatment, including scheduling, path failure detection and path initiation.
    Path capability assessment ensures the tunnel's integrity by deciding the role of the path in the tunnel, since paths can differ in term of capability (throughput, packet loss rate, ...).
    \item Provide support for flow-level control and treatment, including setting and enforcing priority, traffic shaping, resource allocation for sensitive requirement (i.e path for low latency data).
    \item Provide a centralized controller.
\end{itemize}

\subsection{Nonfunctional requirements}
Besides the reliability's \ac{QoS} characteristic, we define the metrics to measure performance of a tunnel as follow:
\begin{itemize}
    \item Bandwidth: Quantity of data that can be transmitted through per second, usually measured in Megabit per Second (Mb/s) or Gigabit  per Second (Gb/s)
    \item Latency: Delay, measured in microsecond (us) or millisecond (ms). Latency introduced by the tunneling process at both ends compared to sending data over the same public path without the tunnel.
    \item System load: Tunneling process costs CPU time and RAM usage.
\end{itemize}

The non-functional requirements for the \ac{MTX} library:
\begin{itemize}
    \item Usability: \ac{MTX} library should be used similar to a Linux socket. A global configuration will be responsible for managing the overall configuration of the tunnel. On the other hand, each caller will have the ability to configure the traffic characteristics according to their specific requirements.
    \item Performance and Reliability: The tunnel is expected to deliver throughput asymptotic to the line rate while having very low negative effect on overall system load thanks to \ac{xdppage} technology. Measuring is rather complicated since the code quality can't be ensured and high performance hardware will be needed. This problem should be addressed in the course of the project.
\end{itemize}
