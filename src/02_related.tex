\cleardoublepage\chapter{Related Work}\label{sec:related_work}\minitoc\vspace{.5cm}\index{SotA}

\section{Introduction}

% \begin{wrapfigure}{r}{0.2\textwidth}
%     \centering
%     \includegraphics[width=0.2\textwidth]{resources/images/example3}
% \end{wrapfigure}

% \sidenote{Overview}
% \todomid{write}

The concept of multipath is well-established, and in this section, we will explore notable existing multipath protocols, their characteristics, and highlight the unique aspects of our concept. 
Additionally, we will discuss future use cases for our \ac{MTX} and examine current solutions and technologies being utilized. 
Finally, we will take a closer look at the design of the 5G system, with a specific focus on the Data Plane, to explain our plan for showcasing the multipath tunnel using the \ac{gptu} protocol.

\section{Tunneling}\index{Tunneling}
\todo{these following are my words, but do I need cite here?}
The concept of a tunnel draws inspiration from its real-world counterpart, which refers to a confined pathway constructed to direct transportation between two points, separate from the surrounding terrain. 
In the physical sense, tunnels are often excavated deep beneath the earth's surface to bypass obstacles like mountains, rivers, or ocean channels.
In the realm of networking, a tunnel can be perceived as a direct network pathway that enables access to another network from the network to which it is currently connected. 
From an infrastructure standpoint, a tunnel serves as an abstraction layer that abstracts the actual traffic routing from the application and isolates the tunnel's traffic from public network traffic. 
This is typically achieved by encapsulating the application's data packets within the tunnel's packets, and then extracting the original data at the other end of the tunnel.
Many tunnel protocols exist for different purposes: IPv4/IPv6 tunnels to enable compatibility between exclusive networks \cite{rfc4380_Teredo_ipv6_tunnel_udp}, Secure Shell (SSH) - tunnel for remote access and data transfer \cite{rfc4251_ssh_protocol}, and \ac{VPN} tunneling - used for access other network from another network. 
\ac{VPN}s are frequently utilized to remotely access a private network, such as connecting to university's network from home. 
It also functions to conceal the origin of network traffic by making it appear as if it originates from the VPN server, which is useful for bypassing firewalls or restrictions applied by local network providers.
The traffic can also be encrypted to add an extra layer of protection while accessing networks over untrusted connections, like public WLAN provided by coffee shops.
There are several widely recognized VPN protocols, including WireGuard, OpenVPN, IPsec, Cisco AnyConnect VPN, L2TP/IPsec, SSTP (Secure Socket Tunneling Protocol), ... each embodies its unique philosophy, background, and intended field of application.


\section{Multipath Connection}\index{Multipath Connection}\label{sec:related_work:mp_connection}
Our primary emphasis is on establishing a connection between two points, irrespective of the number of physical transportation links.
% We will explore various protocols that implement the concept and make comparisons with our approach.
There are 2 approaches to utilize multiple links: managing multiple underlying connections created by existing protocols (\ac{MPTCP} with TCP), and creating a new multipath protocol \ac{SCTP}.
Drawing inspiration from these protocols, our design of \ac{MTX} adopts a hybrid approach: a tunnel that transfers data over multiple UDP connections instead of maintaining TCP sessions. 
The tunnel comprises two primary logical components: the control plane and the transport plane. 
The control plane handles session management, congestion control overhead, and other administrative tasks, enabling the transport plane to be constructed with wrappers over UDP sockets.
Unlike serving solely the socket caller, the multipath tunnel is designed to offer connection and flow management for multiple applications, enabling centralized resource management and scheduling.

\subsection{Multipath TCP}\label{sec:related_work:mp_connection:MPTCP}
% MultiPath TCP (MPTCP) is an effort towards enabling the simultaneous use of several IP-addresses/interfaces by a modification of TCP that presents a regular TCP interface to applications, while in fact spreading data across several subflows. Benefits of this include better resource utilization, better throughput and smoother reaction to failures. 

\ac{MPTCP} is is a major extension to TCP to decoupling TCP from the transport layer by utilizing multiple sub flows, which are underlying TCP connections \cite{Bonaventure_mptcp_decoupling}.
The protocol is showcased in both mobile and data center environments, serving distinct objectives. 
In addition to enhancing data center performance by employing \ac{MPTCP}'s distributed load-balancing across multiple paths, the protocol also provides redundancy and effective congestion management capabilities \cite{raiciu_improving_nodate}.
In the context of mobile hand-over capability, \ac{MPTCP} enhances the ability to maintain TCP sessions while the user equipment moves continuously. 
For instance, a smartphone can continue utilizing both LTE and Wi-Fi connections for a TCP session and predict the optimal route for transmitting most of the traffic. 
By monitoring the radio signal strength, it becomes possible to anticipate potential disruptions in the Wi-Fi connection or when a user enters a building and encounters a loss of cellular signal. 
The author argues that although maintaining two radios simultaneously consumes more power, it can potentially reduce the time required to establish new connections and transmit data \cite{paasch_multipath_2014}.

\begin{figure}[H]
	\centering
	\includegraphics[width=1.0\textwidth]{resources/images/3G_WiFi_Handover_with_Multipath_TCP.PNG}
	\caption{3G/WiFi Handover with Multipath TCP \cite{paasch_multipath_2014}. The MPTCP connection persists despite of WiFi and 3G's unreliable connections. A weak WiFi signal can be indicative of the device moving beyond the coverage range of the router. The command \textit{REXMIT} changes the time-out value for each packet which is used for retransmitting packet.}
    \label{fig:related_work:3G_WiFi_Handover_with_Multipath_TCP}
\end{figure}

\subsection{Stream Control Transmission Protocol}\label{sec:related_work:mp_connection:SCTP}
\ac{SCTP} introduces multiple addresses to the transport layer, which serves as failover and simultaneous underlying connection.
Unlike \ac{MPTCP}, existing internet's infrastructure such as firewalls, routers were not designed to handle \ac{SCTP}'s packets and thus severely limits usage of the protocol \cite{paasch_multipath_2014}. 
Notably, \ac{SCTP} is used in 5G core design for transmitting messages in Control plane.

\begin{figure}[H]
	\centering
	\includegraphics[width=0.8\textwidth]{resources/images/3gpp_5g_part_of_control_plane_protocol.png}
	\caption{Control Plane protocol stack between the UE, the 5G-AN, the AMF and the SMF \cite{3gpp_5g_system_overview}}
    \label{fig:related_work:3gpp_5g_part_of_control_plane_protocol}
\end{figure}


\section{5G Deployment}\index{5G Deployment}
5G services and functions are designed to be deployed in containers and virtual machines, often on server-grade general purpose computers.

These components are categorized into two groups: the Control plane and the Data plane.

As mentioned in \Cref{sec:related_work:mp_connection:SCTP}, \ac{SCTP} protocol with multipath support is used in control plane.
The Data plane relies on \ac{gptu} protocol with \ac{UDP} as the transport protocol to exchange high volume of data between \ac{UPF} and \ac{DN}

\begin{figure}[H]
	\centering
	\includegraphics[width=1.0\textwidth]{resources/images/Non_Roaming_5G_System_Architecture_in_reference_point_representation.png}
	\caption{Non-Roaming 5G System Architecture in reference point representation \cite{3gpp_5g_system_architect_spec_release_18}}
    \label{fig:related_work:Non_Roaming_5G_System_Architecture_in_reference_point_representation}
\end{figure}



% \begin{figure}[H]
%     \centering
%     \includegraphics[width=.55\textwidth]{resources/images/example3}
%     \caption{Related area 1 within the structure of research}\label{fig:hourglass:ra1}
% \end{figure}

% \sidenote{Overview}
% \todomid{write about \Cref{fig:hourglass:ra1}}

% \sidenote{Focus}
% \todomid{write}

% \subsection{Specific Example 1}

% \sidenote{Definition}
% \todomid{write}

% \sidenote{Issues}
% \todomid{write}

% \subsection{Specific Example 2}

% \sidenote{Definition}
% \todomid{write}

% \sidenote{Implementations}
% \todomid{write}

% \sidenote{Research}
% \todomid{write}

% \sidenote{Standards}
% \todomid{write}

% \sidenote{Adoption}
% \todomid{write}

% \subsection{Specific Example 3}\index{Example 3}

% \sidenote{Transition}
% \todomid{write about \Cref{fig:sota:trans}}

% \begin{figure}
%     \centering
%     \includegraphics[width=.85\textwidth]{resources/images/example3}
%     \caption{Comparison of Example 2 and Example 3 (based on~\cite{li2002design})}\label{fig:sota:trans}
% \end{figure}

% \sidenote{Standards}
% \todomid{write}

% \sidenote{Extension}
% \todomid{write}

% \sidenote{Other Standards}
% \todomid{write}

% \sidenote{Something}
% \todomid{write}

% \sidenote{Something}
% \todomid{write}

% \sidenote{Something}
% \todomid{write}

% \sidenote{Something}
% \todomid{write}

% \sidenote{Something}
% \todomid{write}

% \sidenote{Something}
% \todomid{write}

% \section{Related Area 2}\index{Related Area 2}

% \sidenote{Overview}
% \todomid{write}

% \sidenote{Focus}
% \todomid{write about \Cref{fig:sota:ra2}}

% \begin{figure}[!hbtp]
%     \centering
%     \includegraphics[width=1\textwidth]{resources/images/example3}
%     \caption{Related Area 2}\label{fig:sota:ra2}
% \end{figure}

% \sidenote{Something}
% \todomid{write}

% \subsection{Specific Example 1}

% \sidenote{Definition}
% \todomid{write}

% \sidenote{Issues}
% \todomid{write}

% \subsection{Specific Example 2}

% \sidenote{Definition}
% \todomid{write}

% \sidenote{Implementations}
% \todomid{write}

% \sidenote{Research}
% \todomid{write}

% \sidenote{Standards}
% \todomid{write}

% \sidenote{Adoption}
% \todomid{write}


% \section{Related Area 3}\index{Related Area 3}

% \sidenote{Overview}
% \todomid{write}

% \sidenote{Focus}
% \todomid{write about \Cref{fig:sota:ra3}}

% \begin{figure}[!hbtp]
%     \centering
%     \includegraphics[width=1\textwidth]{resources/images/example3}
%     \caption{Related Area 3}\label{fig:sota:ra3}
% \end{figure}

% \sidenote{Something}
% \todomid{write}

% \subsection{Specific Example 1}

% \sidenote{Definition}
% \todomid{write}

% \sidenote{Issues}
% \todomid{write}

% \subsection{Specific Example 2}

% \sidenote{Definition}
% \todomid{write}

% \sidenote{Implementations}
% \todomid{write}

% \sidenote{Research}
% \todomid{write}

% \sidenote{Standards}
% \todomid{write}

% \sidenote{Adoption}
% \todomid{write}

% \section{Conclusion}

% \sidenote{Summary}
% \todomid{write}

% \sidenote{Takeaway 1}
% \todomid{write}

% \sidenote{Takeaway 2}
% \todomid{write}

% \sidenote{Takeaway 3}
% \todomid{write}

% \sidenote{Next chapter}
% \todomid{write}
